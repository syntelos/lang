\documentclass[12pt,twocolumn]{article}

\begin{document}

\title{Introducing logical spacetime}

\author{John D.H. Pritchard \thanks{@syntelos, logicalexistentialism@gmail.com}}

\date{\today}

\maketitle

%\tableofcontents

\begin{abstract}

Semiconductor system design would benefit from a spacetime foundation.
A nanosecond spans a centimeter with a common relativistic frame.  The
individual inertial reference frames of each electron in the square
centimeter of an integrated system are logically coherent to a
nanosecond ``clop''.
  
\end{abstract}


\section{Logical reference frame}

A VLSI paradigm assures the coherency of an integrated logical
spacetime.  The inertial reference frames of individual electrons are
not temporal as independent.  They are not related
individualistically.  However, their number and distance has a
temporal relation.  And that relationship has a degree of
individualism.  The paradigm must assure temporal coherence.

An operational clock cycle (``clop'') is the principal concept of
unification.  The integration of relativistic particle-waves to the
performance of work requires a coherent conception.  The {\it clop}
enables that coherence by introducing temporal quantization to the
physical reference frame.  


%\appendix

%\bibliographystyle{plain}
%\bibliography{mst}

%\section{Notes}



\end{document}
