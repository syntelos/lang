\documentclass[12pt,twocolumn]{article}

\begin{document}

\title{Radiant object applications}

\author{John D.H. Pritchard \thanks{@syntelos, logicalexistentialism@gmail.com}}

\date{\today}

\maketitle

%\tableofcontents

\begin{abstract}

Radiant object theory \cite{ROTI} and topology \cite{ROTY} opened a
perspective on social peer trust networks.  This effort is an
examination of the application of those networks as radiant package
\cite{RFC5050} systems.
  
\end{abstract}


\section{Structure}

Peer networks serve highly constrained (mobile) devices in a highly
constrained spacetime.  Space and time are both scarce to peer
applications.  When the {\bf package that asserts a demand request}

$$
 \pi_D
$$

and the {\bf package that defines a supply response}

$$
 \pi_S
$$

are minimized to the application semantics that have reduced an
application domain to an {\bf application domain representation and
  operation},

$$
 \alpha_{\rho + \lambda}
$$

the spacetime resources have been reasonably conserved.

$$
 \pi^{\alpha}_{D} \longleftrightarrow \pi^{\alpha}_{S}
$$

Each peer node endpoint that has a copy of \(\pi^{\alpha}_{S}\) may be
a distributed supply endpoint, acccording to the evaluation of
\(\pi^{\alpha}_{D}\).

$$
 \pi^{\alpha}_S \leftarrow \prod \pi^{\alpha}_{D}
$$

The ephemeral application processing framework determines the
boundaries and character of the evaluation of \(\pi^{\alpha}_{D}\).

\section{Possible worlds}

A relatively implicit context is the world of the web browser.  The
HTTP \cite{RFC2616} request and response message pair are subsumed by
an application context, as well as containing streams.  A relatively
explicit context removes the request and response message pair from
contextual dependencies.  In this case the domain of origin includes
independent demand processing.  Original interdependence should be
well defined, well known, and readily reproducible.

When this is true, the evalutation

$$
 \prod \pi^{*}_{D}
$$

has a standard meaning as includes the success or failure of an
intermediary to supply distributed content.

Standard human interaction technologies have been explored in the W3C
\cite{HTML} and IETF \cite{RFC2045}.  The span from the necessities of
representation to the demands upon representation envelops many
worlds, from character codes and portable bitmaps to virtual reality.
The communication domain supports the original context as developed in
the ITU \cite{X400}.  A character code may be negotiated or
discovered.  Likewise a human computer interaction application (HCI/A)
framework (HCI/A/F).

The platform is not generic.  The peer endpoint device may be mobile
or immobile without relevance to the distinction between platform and
framework.  The conception of HCI/A/F as independent of platform has
been blurred by varieties of the conception of platform.  Some
``platforms'' are ``frameworks'' as available to specification as
HCI/A media format.  The Scalable Vector Graphics (SVG) \cite{SVG} is
a complete HCI/A/F when it embeds local interaction via JavaScript
\cite{JS}.  And some platforms are functionally equivalent to SVG.

\appendix

\bibliographystyle{plain}
\bibliography{rot}

\end{document}
