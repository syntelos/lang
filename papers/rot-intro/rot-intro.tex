\documentclass[12pt,twocolumn]{article}

\begin{document}

\title{Introducing radiant object theory}

\author{John D.H. Pritchard \thanks{@syntelos, logicalexistentialism@gmail.com}}

\date{\today}

\maketitle

%\tableofcontents

\begin{abstract}

The development of peer media might benefit from a treatment of
distributed object application spaces.  The perspective offered is not
a general review of peer media.  The abstraction offered is intended
to motivate development.
  
\end{abstract}


\section{Background}

In the X.400 \cite{X400} paradigm the ITU established a static binary
form to communicate well defined messages.  In the MIME \cite{RFC2045}
paradigm then IETF established a textual form to communicate
transparently in a message object finitude.  In the first case we find
the conservation of definition.  And in the second case we find the
conservation of transparency.

To conserve transparency in the textual case of MIME standards is to
contrast and compare fabrication, production, and consumption
processes.  The obscurity that X.400 products fall prey to is an
opacity due to tooling.  The complexity that MIME products are victims
of is alternative.

Neither has avoided interference in the affairs of communication.
However, fidelity to human ideals of transparency and facility is well
served by the IETF simplicity.  With textual transparency and
technical reproducibility the retreat to obscurity is elided.

These are matters of practice.  How establishment evolves to the
sustenance of humanity or the exploitation of inhumanity.  The ITU
paradigm was by far the more perfect establishment, however that
establishment has allowed those retreats to obscurity that defeat
technical reprodicibility.

\section{Theory}

The matters of theory are spatial and temporal.  The spatial logics of
digital resources and communication, and the temporal logics of
spatial objects and processes.  Spatially systemic and universal.
Temporally immediate and remote.

When a spatial communication distinction is terrestrial or
interplanetary, the systemic spatial interest becomes consumption.
When extreme consumption is not a focus of concern, the textual
reproduction standards become interesting.  This may be the case of
peer media.

\section{Structure}

The peer media as inclusive of mobile devices is characterized by the
contact internetworking of trust relationships.  In this conception,
the structure of applications is {\it ad hoc}, and the distribution of
objects as compose dictionaries of name spaces is not ordinary but
partial.  It is the identity of individuals and automata and
credentials that are unique by proof.

The {\it partial object distribution} \((\omega\delta)\) over {\it trust relational connectivity} \((\tau\kappa)\) 

 $$
 \frac{\omega\delta}{\tau\kappa}
 $$

represents a class of peer media in the spatio-temporal logic of
information and communication.  In comparison, the ordinary or {\it
  probable object distribution} \((\pi\delta)\) 

 $$
 \frac{\pi\delta}{\tau\kappa}
 $$

integrates reachability and distribution as objective.

 $$
 \frac{\omega\delta}{\tau\kappa} \le \frac{\pi\delta}{\tau\kappa}
 $$



\appendix

\bibliographystyle{plain}
\bibliography{rot}

%\section{Notes}



\end{document}
